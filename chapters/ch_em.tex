\chapter{Electricity \& Magnetism}

\section{Maxwell's Equations}
\begin{align}
	\nabla\cdot\mathbf{E}&=4\pi\rho \label{eq:gauss}\\[0.5em]
	\nabla\cdot\mathbf{B}&=0 \label{eq:divB0}\\[0.5em]
	\nabla\times\mathbf{E}&=-\frac{1}{c}\frac{\partial\mathbf{B}}{\partial t} \label{eq:faraday}\\[0.5em]
	\nabla\times\mathbf{B}&=\frac{4\pi}{c}\mathbf{j} + \frac{1}{c}\frac{\partial\mathbf{E}}{\partial t} \label{eq:ampere_maxwell}
\end{align}

\section{Basic Properties of Maxwell's Equations}
\begin{itemize}
	\item Eq. \ref{eq:gauss} relates the flux of the electric field through any ``Gaussian surface'' to the total enclosed charge. By counting the field lines through the surface, we may have an idea of the total enclosed charge, or the charge distribution.
	\item Eq. \ref{eq:divB0} says that there are no magnetic monopoles. Unlike electric fields, there are no ``magnetic point charges'' to generate magnetic fields. Rather, a magnetic field must be generated by the \textit{flow} of charge, i.e. a current. This fields are of course also generated by magnetic dipoles which may be viewed as miniature current loops. More specifically, this statement says that any magnetic field line that leaves a given volume must return to that given volume such that the net flux is zero. 
	\item Eq. \ref{eq:faraday} says that a time varying magnetic field can produce an electric field. This is a statement of electromagnetic induction.
	\item Eq. \ref{eq:ampere_maxwell} says that a magnetic field can be generated both by a current and by a time-varying electric field.
	\item Because $\nabla\cdot\mathbf{B}=0$, we may express $\mathbf{B}$ as the curl of some vector field $\mathbf{A}$ such that $\nabla\times\mathbf{A}=\mathbf{B}$, where $\mathbf{A}$ is referred to as the magnetic vector potential. Using Eq. \ref{eq:faraday}, we find that $\nabla\times\left(\mathbf{E} + \frac{1}{c}\frac{\partial}{\partial t}\mathbf{A}\right) = 0$. Thus, we may express the term in the parentheses as $-\nabla\phi$, the gradient of a vector potential. This leads to the familiar formulation.
	\begin{equation}
		\mathbf{E} = -\nabla\phi - \frac{1}{c}\frac{\partial }{\partial t}\mathbf{A}.
	\end{equation}
\end{itemize}
%
\section{Electrostatics}
\section{Magnetostatics}
\section{Boundary Value Problems}
\section{Waves}
